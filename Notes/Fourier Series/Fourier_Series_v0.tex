% ================
% Landon Buell
% Kevin Short
% PHYS 799 
% 5 May 2020
% ================

\documentclass[12pt,letterpaper]{article}

\usepackage{amsmath}
\usepackage{amssymb}
\usepackage{multicol}
\usepackage[left=2.5cm,right=2.5cm,top=2.5cm]{geometry}
\usepackage{float}
\usepackage{graphicx}
\usepackage{fancyhdr}

% ================================================================

\pagestyle{fancy}
\fancyhf{}
\lhead{An Introduction to Fourier Analysis}
\rhead{Landon Buell}
\cfoot{\thepage}

% ================================================================

\begin{document}

\title{An Introduction to Fourier Analysis}
\author{Landon Buell}
\date{5 May 2020}
\maketitle

% ================================================================

\section*{Introduction}
\paragraph*{}Just after the turn of the 18-th century French Mathematician Jean Baptiste Fourier had spent a great deal of his time in the subjects of various partial differential equations and boundary value problems. In particular, he found that analytical solutions to \textit{The Heat Transfer Equation} involved an infinite series of orthogonal trigonometric functions \cite{Pinsky,Olver}. This lead Fourier to extent his findings and propose that \textit{any} function that was periodic with space or time could similarly be expressed as this series of sine and cosine functions. For this work, we will take explore time-domain related derivation, behavior, and analysis of the properties of and related to Fourier Series, Analysis and Transforms, both analytically and computationally.

% ================================================================


\section*{Fourier Series Definition}
\paragraph{}Since the series revolves around the idea of converging around approximations of period functions, let us define a continuous function, 
$f$, that is periodic in time-space, to have the property \cite{Tolstov}:
\begin{equation}
\label{periodic definition}
f(t) = f(t + nT)
\end{equation}
Where $T$ is the \textit{period} of the function with units of time ($T \neq 0$), and $n$ is an integer to indicate that the function repeats for any integer $n$. More intuitively, this is function when evaluated at time $t$ takes on some value, and then will return to that same value when progressing forward or backward in time, by some integer multiple of the period. 
\paragraph*{}With this distinction made, we can examine Fourier's claim that we can approximate this function, over some length of time, $L$, as a sum of sine and cosine functions \cite{Pinsky,Tolstov}. Mathematically:
\begin{equation}
\label{Series Def}
f(t) = a_0 + \sum_{n=1}^{+\infty} \Bigg[ a_n \cos\Big(\frac{n\pi t}{L}\Big) + b_n \sin\Big(\frac{n\pi t}{L}\Big) \Bigg]
\end{equation}
Were $a_0$ , $a_n$ and $b_n$ are coefficients $L$ is the bound of time which we are concerned with and $n$ is an integer that indicates that we want our analysis to include integer multiples of unit frequencies.
\paragraph*{}This serves as a sort of starting point or baseline for most of Fourier Analysis. If we can accept, or at least humor this, idea, we can use this foundation to break some sort of function, $f(t)$ down into a collection of functions of a single frequency. To do this more practically, we need to derive an expression for each of the infinitely many $a_n$ and $b_n$ coefficients. In order to construct expression for them, we require three very important properties of trigonometric functions:

\begin{enumerate}
\item[•]\textbf{Orthogonality of Cosine}
\begin{equation}
\label{cos orth}
\int_{0}^{L} \cos\Big(\frac{n\pi t}{L}\Big) \cos\Big(\frac{m\pi t}{L}\Big) dt = 
	\left\{
        \begin{array}{ll}
            0 	&, n \neq m \\
            2/L	&, n = m \neq 0 \\
            L 	&, n = m = 0\\
        \end{array}
    \right.
\end{equation}
\item[•]\textbf{Orthogonality of Sine}
\begin{equation}
\label{sin orth}
\int_{0}^{L} \sin\Big(\frac{n\pi t}{L}\Big) \sin\Big(\frac{m\pi t}{L}\Big) dt = 
	\left\{
        \begin{array}{ll}
            0 	&, n \neq m \\
            2/L	&, n = m \neq 0 \\
            L 	&, n = m = 0\\
        \end{array}
    \right.
\end{equation}
\item[•]\textbf{Mutual Orthogonality}
\begin{equation}
\label{mutual orth}
\int_{0}^{L} \cos\Big(\frac{n\pi t}{L}\Big) \sin\Big(\frac{m\pi t}{L}\Big) dt = 0 ,
\hspace*{1cm} \forall m,n
\end{equation}
Note that is last property holds in $m$ or $n$ are $0$, which causes that function to oscillate with $0$ frequency, this remain constant.
\end{enumerate}
\paragraph*{}Using these three identities, we can show uncover the coefficients.


% ================================================================

% ================================================================


% ================================================================


% ================================================================


% ================================================================

\begin{thebibliography}{9}

\bibitem{Debnath}
Debnath, Lokenath, and Dambaru Bhatta. \textit{Integral Transforms and Their Applications}. CRC Press/Taylor \& Francis Group, 2015.

\bibitem{Haberman}
Haberman, Richard. \textit{Applied Partial Differential Equations: with Fourier Series and Boundary Value Problems}. 5th ed., Pearson Education, 2014.

\bibitem{Olver}
Olver, Peter J. Introduction to Partial Differential Equations. Springer International, 2016.

\bibitem{Oppenheim}
Oppenheim, Alan S., et al. \textit{Signals and Systems}. Prentice-Hall, 1983.

\bibitem{Peatross}
Peatross, Justin, and Michael Ware. \textit{Physics of Light and Optics}. Brigham Young University, Department of Physics, 2015.

\bibitem{Pinsky}
Pinsky, Mark A. \textit{Partial Differential Equations with Boundary-Value Problems with Applications}. 3rd ed., American Mathematical Society, 2010.

\bibitem{Tolstov}
Tolstov, Georgy Pavlovich. \textit{Fourier Series}. Translated by Richard Allan. Silverman, London; Printed in U.S.A., 1962.

\end{thebibliography}


% ================================================================


\end{document}