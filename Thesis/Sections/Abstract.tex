% ================================
% Landon Buell
% Kevin Short
% Physics 799
% 5 September 2020 
% ================================


\documentclass[12pt,letterpaper]{article}
\usepackage{amsmath}
\usepackage{amssymb}
\usepackage{float}
\usepackage{graphicx}
\usepackage{xcolor}
\usepackage[top=2.5cm,left=2.5cm,right=2.5cm]{geometry}

\begin{document}

% ================================================================

\section{Abstract}

\paragraph*{}Machine learning has been a major player in the role of audio processing and classification for decades. In this time, a great deal of work has been devoted to studying the performance of various model architectures and producing sets of features that can represent a waveform in a compact, efficient, and non-redundant way \textcolor{red}{Citation?}. In this work, we explore a hybrid convolutional neural network (CNN) and multilayer perceptron (MLP) model that uses  features derived from the time-series and frequency-series of a waveform to map audio to a potential source. We examine each features used in classification as well as the physical significance and how the choice of feature influences the neural network architecture. In doing so, we show how the appropriate combination of architecture and features can produce a reasonable classification performance when mapping sound files to a potential musical instrument source.

% ================================================================

\end{document}